\chapter{Introduction}

% TODO cases of forged images in journalism and science?

% TODO how long has the field been around?

% TODO examples of research


Image forensics, the study of finding hidden information in photographs, is an emerging field with diverse applications. It is comprised of a few general subtopics: determining the source of an image, detecting forgery, and detecting steganography (hiding information in a digital image) \cite{rocha2011}. All too often, pictures are used as hard evidence in journalism, scientific research, and criminal cases. For example, there was one incident in the scientific community in which a group of South Korean scientists published groundbreaking stem cell research results, but it was later revealed that the majority of the photographs included with the results were doctored \cite{rocha2011}. Image forgery itself predates the digital age, but as computers have become more capable and tools such as Adobe Photoshop have been developed, it has become much easier to manipulate images in a convincing manner. Therefore, there is an increasing need for forensic techniques as the technology develops.

Existing forensic techniques generally involve machine learning, finding statistical anomalies, or finding geometric inconsistencies. The current techniques that take advantage of reflective objects in an image fall into the latter category. Interestingly, the human eye is very insensitive to inconsistencies in reflections \cite{farid2010image}. Therefore it would be fairly straightforward for a forger to, for example, convincingly show a person in a scene with a reflective surface who was never actually there. It would perhaps be convincing to a human, but may not hold up to analysis with a geometric technique.

Most of the reflective surfaces that might be found in real photos are planar. Some common examples are bodies of still water, mirrors, and glass windows. Non-planar reflective surfaces, which are the focus of this paper, are less common but can be found in the form of store security mirrors, works of art, and even the human eye.