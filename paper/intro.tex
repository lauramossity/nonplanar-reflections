\chapter{Introduction}

% TODO cases of forged images in journalism and science?

% TODO how long has the field been around?

% TODO examples of research

% TODO how sensitive are humans to seeing inconsistent reflections? (see obrien12)

Image forensics, the study of finding hidden information in photographs, is an emerging field with diverse applications. It is comprised of a few general subtopics: determining the source of an image, detecting forgery, and detecting steganography (hiding information in a digital image) \cite{rocha2011}. All too often, pictures are used as hard evidence in journalism, scientific research, and criminal cases. Image forgery is nothing new, but as computers have become more capable and tools such as Adobe Photoshop have been developed, it has become much easier to manipulate images in a convincing manner. Therefore, there is a need for forensic techniques.

Existing forensic techniques generally involve finding statistical anomalies, machine learning, or finding geometric inconsistencies. The current techniques that take advantage of reflective objects in an image fall into the latter category.

Most of the reflective surfaces that might be found in real photos are planar. Some common examples are bodies of still water, mirrors, and glass windows. Non-planar reflective surfaces are less common, but can be found in the form of store security mirrors, works of art, and even the human eye.